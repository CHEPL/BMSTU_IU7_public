\documentclass[a4paper,12pt]{article}

\usepackage[english, russian]{babel}

\usepackage[pdftex,
            pdfauthor={Андрей Ч.},
            pdftitle={лр1 лр2 моделирование},
            pdfsubject={Математическое моделирование}]{hyperref}

\AfterPreamble{\hypersetup{
  pdfauthor={Андрей Ч.},
  pdftitle={лр1 лр2 моделирование},
  pdfsubject={Математическое моделирование}
}}

\usepackage{amsmath}

\usepackage{xcolor}

\usepackage{pgf}

\usepackage{subcaption}

\usepackage{graphicx}
\usepackage{grffile}

\usepackage{float}

\usepackage{geometry}
\geometry{left=2cm}
\geometry{right=1.5cm}
\geometry{top=1cm}
\geometry{bottom=2cm}

\begin{document}

    \section*{Общее для ЛР2 и ЛР3}

    Постановка задачи:

    Цилинр радиуса R заполнен изулучающим газом, 
    температурное поле T(r) задано.

    $z=\frac{r}{R}$ - безразмерная независимая величина, $r$ - радиус слоя, для 
    которого находим зависимые величины.

    Требуется вычислить поток излучения $F(z)$ и объёмную плотность излучения $u(z)$
    на промежутке $z={0,1}$

    \begin{figure}[H]
        %\includegraphics[width=.4\linewidth]{\input{k1.pgf}}
        %\resizebox{\textwidth}{!}{\input{visual.pdf}}
        \includegraphics[width=\linewidth]{visual.pdf}
        \caption{Визуализация задачи}
    \end{figure}

    По смыслу задачи:

    $F(z)$ - поток энергии через слой (от оси к стенке цилиндра);

    $u(z)$ - объёмная плотность энергии излучения в слое.

    \vspace{1cm}

    Не смотря на размерности потока и объёмной плотности ($\frac{\text{Вт}}{\text{см}^2}$
    , $\frac{\text{Дж}}{\text{см}^3}$)
    , задача одномерная
    (решается относительно z - одной независимой переменной, вдоль радиуса
    поперечного сечения цилидра).

    \newpage
    Исходная система уравнений:

    \Large
    \begin{equation}
        \begin{cases}
            F=-\frac{c}{3Rk(T(z))}\frac{du}{dz}\\
            \frac{1}{zR}\frac{d}{dz}(zF) = ck(T(z))(u_p - u)\\
            T(z)=(T_{w}-T_{0})*z^p + T_{0}\\
            u_p(z)=\frac{0.0003084}{e^{\frac{47990}{T(z)}}-1}
        \end{cases}
    \end{equation}
    \normalsize

    \vspace{1cm}
    Начальные условия:

    \Large
    \begin{equation}
        \begin{aligned}
            &z=0,F(0)=0\\
            &z=1,F(1)=0.393cu(1)\\
            &T_{w}=2000K\\
            &T_{0}=10000K\\
            &R=0.0035m\\
            &c=299792458\frac{m}{s^2}\\
            &p=4
        \end{aligned}
    \end{equation}
    \normalsize

    \vspace{1cm}
    Замены:

    \Large
    \begin{equation}
        \begin{aligned}
            &a(z)=-\frac{c}{3Rk(z)}\\
            &b(z)=cRk(z)
        \end{aligned}
    \end{equation}
    \normalsize

    \vspace{1cm}
    Условности нотации:

    \Large
    \begin{equation}
        \begin{aligned}
            &k(z)\equiv k(T(z))\\
            &z_n = h * n , h - \text{шаг численного метода}\\
            &a(z_n) \equiv a_n, (\text{аналогично для } b(z), k(z), u_p(z), F(z), u(z))
        \end{aligned}
    \end{equation}
    \normalsize
    

    \newpage
    $k(T)$ выглядит следующим образом:

    %\begin{figure}[!h]
    %    \centering
    %    \resizebox{0.8\textwidth}{!}{\input{k1.pgf}}
    %\end{figure}

    \begin{figure}[H]
        \centering
        \begin{subfigure}{.5\textwidth}
          \centering
          %\includegraphics[width=.4\linewidth]{\input{k1.pgf}}
          \resizebox{\textwidth}{!}{\input{k1.pgf}}
          \caption{k(T) - Вариант 1}
          \label{fig:sub1}
        \end{subfigure}%
        \begin{subfigure}{.5\textwidth}
          \centering
          %\includegraphics[width=.4\linewidth]{\input{k2.pgf}}
          \resizebox{\textwidth}{!}{\input{k2.pgf}}
          \caption{k(T) - Вариант 2}
          \label{fig:sub2}
        \end{subfigure}
        \label{fig:test}
    \end{figure}


    Если интерполировать как на графике (полиномом 1 степени), то получим некоторую ошибку.

    Примения предложенный в методе метод ($t\_ln=ln(T)$, $k\_ln=ln(k)$), получаем:

    \begin{figure}[H]
        \centering
        \begin{subfigure}{.5\textwidth}
          \centering
          %\includegraphics[width=.4\linewidth]{\input{k1.pgf}}
          \resizebox{\textwidth}{!}{\input{k1_ln.pgf}}
          \caption{k\_ln(t\_ln) - Вариант 1}
          \label{fig:sub1}
        \end{subfigure}%
        \begin{subfigure}{.5\textwidth}
          \centering
          %\includegraphics[width=.4\linewidth]{\input{k2.pgf}}
          \resizebox{\textwidth}{!}{\input{k2_ln.pgf}}
          \caption{k\_ln(t\_ln) - Вариант 2}
          \label{fig:sub2}
        \end{subfigure}
        \label{fig:test}
    \end{figure}

    Ошибка ожидается около-нулевой, $k(T)$ вычисляется следующим образом:

    $k(T) = e^{c_{1} * ln(T) + c_{0}}$

    \small
    *$c_{1}$ и $c_{0}$ можно посчитать руками, решив систему уравнений 1 степени для крайних значний
    $k(2000)$, $k(10000)$. Значния полученные с помощью numpy.polyfit: 
    $c_{1}=2.99996105$ , $c_{0}=-27.60599153$ - Вариант 1,
    $c_{1}=3.0$ , $c_{0}=-22.33270375$ - Вариант 2.
    \normalsize

    \newpage
    \section*{ЛР2}

    Дифференцируем $zF$ : $\frac{d}{dz}(zF)=F+z*\frac{dF}{dz}$

    Получаем:

    \Large
    \begin{equation}
        \begin{cases}
            \frac{du}{dz}=\frac{F}{a(z)}\\
            \frac{dF}{dz}=b(z)(u_p - u)-\frac{F}{z}
        \end{cases}
    \end{equation}
    \normalsize

    Данная задача решается как задача Коши, если задать начальное условие для $u$:

    \begin{equation*}
        u(0)=\mathcal{E}u_p(0)
    \end{equation*}

    Ожидая, что решение может быть найдено при начальном $\mathcal{E}\in\{0..1\}$
    и что $u(1)$ монотонно зависит от $\mathcal{E}$,
    находим решение методом стрельбы + дихотомии:

    \begin{verbatim}
        Epsilon1 = 0.0
        Epsilon2 = 1.0
        Err = inf

        цикл пока fabs(Err) > заданная точность

            Epsilon = (Epsilon1 + Epsilon2) / 2
            Err = Рунге-Кутта(Epsilon)

            если Err > 0.0
                Epsilon1 = Epsilon
            иначе
                Epsilon2 = Epsilon
            конец если

        конец цикла
    \end{verbatim}

    При этом $Err = \frac{F^{(N)}(1) - 0.393cu^{(N)}(1)}{0.393cu^{(N)}(1)} =
    \frac{F^{(N)}(1)}{0.393cu^{(N)}(1)} - 1$

    Вычисления завершаются на N-ой итерации, когда для очередного приближения 
    $F^{(N)}(z)$, $u^{(N)}(z)$ выполняется
    $\left\lvert Err \right\rvert<= \delta$, 
    где $\delta$ - заданная точность.

    \newpage
    Для решения задачи Коши используем формулы Рунге-Кутты IV порядка точности 
    для двух зависимых переменных:

    \begin{verbatim}

        F = 0
        u = Epsilon * u_p(0)
        z = 0

        пока z <= 1

            P_1 = F / a(z)
            K_1 = b(z) * (u_p(z) - u) - F / z

            z += h / 2

            P_2 = (F + K_1 * h / 2) / a(z)
            K_2 = b(z) * (u_p(z) - u - P_1 * h / 2) - (F + K_1 * h / 2) / z

            P_3 = (F + K_2 * h / 2) / a(z)
            K_3 = b(z) * (u_p(z) - u - P_2 * h / 2) - (F + K_2 * h / 2) / z

            z += h / 2

            P_4 = (F + K_3 * h) / a(z)
            K_4 = b(z) * (u_p(z) - u - P_3 * h) - (F + K_3 * h) / z

            F += (K_1 + 2 * (K_2 + K_3) + K_4) * h / 6
            u += (P_1 + 2 * (P_2 + P_3) + P_4) * h / 6

        конец пока
    \end{verbatim}

    При этом на 1-й итерации при вычислении $K_{1}$ произойдёт деление на 0, поэтому целесообразно
    вынести эту итерацию из цикла и проинициализировать $K_{1}:=b(z) * (u_p(z)-u)$.

    \vspace{2cm}
    *Возможно оптимизировать путем выделения общих для $K_2$ и $K_3$, $P_2$ и $P_3$ частей, а так же
    общей для $a(z), b(z)$ части - $Rk(z)$.

    \newpage
    \subsection*{Вопросы при защите лабораторной работы}

    \textbf{1. Какие способы тестирования программы можете предложить?}

    \color{red}

    Существует вразумительный ответ, но я его забыл))) 

    Точно не стоит предлагать сравнивать результаты вычислений с реальными измерениями, 
    т.к. мы моделируем специально для того, чтобы не экспериментировать ирл.

    Вероятнее всего, речь про тривиальные случаи. Потвикать параметры задачи так, 
    чтобы решение было очевидным, и сравнить с результатом работы программы. 
    Например: задать правое краевое условие $F(1)=100\frac{\text{Вт}}{\text{м}^2}$ и проверить, выполняется ли 
    оно в результате работы программы.

    \color{black}

    \vspace{1cm}
    \textbf{2. Приведите классификацию методов решения систем ОДУ для задачи Коши}

    Аналитические. Можем решить - находим общее решение, используя начальное условие получаем
    частное решение. (такого почти не бывает)

    Приближенно аналитические. Пикар. Если удаётся интегрировать правую часть много раз, 
    получаем сходящийся к точному решению ряд. Сходится плохо. (тоже почти не бывает, но есть
    исключительные задачи по типу $u'(x)=x^2+u^2$, для которых решение чисто аналитически не найти).

    Численные методы: 
    
    \begin{itemize}
        \item одношаговые (для вычисления значений в следующем узле требуется знать значения
        только в предыдущем узле) / многошаговые (для вычисления значений в следующем узле требуется 
        знать значения в нескольких предыдущих узлах).
        \item явные ($u_n = \psi (x_{n-1}, u_{n-1}, f(x_{n-1}, u_{n-1}))$) / неявные
        ($u_n = \psi (x_{n-1}, u_{n-1}, f(x_{n}, u_{n}))$).
        В случае явных следующее значание выражается через предыдущее. В случае неявных
        следующее значание выражается через предыдущее И производную следующего - 
        нельзя применить 'как есть' , нужно искать решение уравнения - не всегда возможно.
        Как правило, неявные методы более устойчивые. Зачастую (всегда, в случае монотонных функций)
        явный и неявный метод сходятся к
        решению с разных сторон, поэтому можно использовать совместно для отыскания интервала, 
        в котором находится точное решение.
    \end{itemize}
    
    \vspace{1cm}
    \textbf{3. Получите систему разностных уравнений для решения сформулированной задачи неявным 
    методом Эйлера. Опишите алгоритм реализации полученных уравнений.}

    Неявный метод Эйлера для двух переменных: 

    \Large
    \begin{equation}
        \begin{cases}
            f_n = f_{n-1} + h f^{'}(x_n, f_n, t_n)\\
            t_n = t_{n-1} + h t^{'}(x_n, t_n, f_n)
        \end{cases}
    \end{equation}
    \normalsize

    Для нашей задачи получаем:

    \Large
    \begin{equation}
        \begin{cases}
            F_n = F_{n-1} + h(b(z_n)({u_p}_n - u_n) - \frac{F_n}{z_n})\\
            u_n = u_{n-1} + h\frac{F_n}{a(z_n)}
        \end{cases}
    \end{equation}
    \normalsize

    %Получаем выражание для $F_n$:
    Подставляем (6.2) в (6.1):

    \Large
    \begin{equation}
        F_n = F_{n-1} + h(b(z_n)({u_p}_n - u_{n-1} - \frac{F_n}{a(z_n)} h) - \frac{F_n}{z_n})\\
    \end{equation}
    \normalsize

    Выражаем $F_n$ из (7):

    \Large
    \begin{equation}
        F_n = \frac{F_{n-1} + hb_n({u_p}_n - u_{n-1})}{1 + \frac{h}{z_n} + \frac{h^2b_n}{a_n}}\\
    \end{equation}
    \normalsize
    
    Система разностных уравнений:

    \Large
    \begin{equation}
        \begin{cases}
            F_n = \frac{F_{n-1} + hb_n({u_p}_n - u_{n-1})}{1 + \frac{h}{z_n} + \frac{h^2b_n}{a_n}}\\
            u_n = u_{n-1} + h\frac{F_n}{a(z_n)}
        \end{cases}
    \end{equation}
    \normalsize

    Алгоритм реализции: 1. Инициализировать $F_0$, $u_0$ 2. Пока $z<=1$ вычисляем $F_n$, $u_n$

    \vspace{1cm}
    \textbf{4. Получите систему разностных уравнений для решения сформулированной задачи 
    неяв-ным методом трапеций. Опишите алгоритм реализации полученных уравнений.}

    Неявный метод трапеций для двух переменных: 

    \Large
    \begin{equation}
        \begin{cases}
            f_n = f_{n-1} + h \frac{f^{'}(x_{n-1}, f_{n-1}, t_{n-1}) + f^{'}(x_n, f_n, t_n)}{2}\\
            t_n = t_{n-1} + h \frac{t^{'}(x_{n-1}, t_{n-1}, f_{n-1}) + t^{'}(x_n, t_n, f_n)}{2}
        \end{cases}
    \end{equation}
    \normalsize

    Для нашей задачи получаем:

    \Large
    \begin{equation}
        \begin{cases}
            F_n = F_{n-1} + h \frac{b(z_{n-1})({u_p}_{n-1} - u_{n-1}) - \frac{F_{n-1}}{z_{n-1}} + b(z_n)({u_p}_n - u_n) - \frac{F_n}{z_n}}{2}\\
            u_n = u_{n-1} + h \frac{\frac{F_{n-1}}{a(z_{n-1})} + \frac{F_n}{a(z_n)}}{2}
        \end{cases}
    \end{equation}
    \normalsize

    Подставляем (11.2) в (11.1) и выражаем $F_n$, получаем систему разностных уравнений:

    \Large
    \begin{equation}
        \begin{cases}
            F_n = \frac{F_{n-1}(1 - \frac{h}{2z_{n-1}} - \frac{h^2b_n}{4a_{n-1}}) + 
            \frac{h}{2}(b_{n-1}{u_p}_{n-1} + b_n{u_p}_n - u_{n-1}(b_{n-1} + b_n))}
            {1 + \frac{h}{2z_n} + \frac{h^2b_n}{4a_n}}\\
            u_n = u_{n-1} + h \frac{\frac{F_{n-1}}{a(z_{n-1})} + \frac{F_n}{a(z_n)}}{2}
        \end{cases}
    \end{equation}
    \normalsize

    Алгоритм реализции: --//-- , поскольку в (12.1) есть деление на $z_{n-1}$, надо на 1-й
    итерации не поделить на 0 (вынести за цикл)

    \vspace{1cm}
    \textbf{5. Из каких соображений проводится выбор численного метода того или иного порядка 
    точности, учитывая, что чем выше порядок точности метода, тем он более сложен и требует, 
    как правило, больших ресурсов вычислительной системы?}

    \color{red}

    Для некоторых задач не является целесообразным использовать методы более высокого порядка
    точности. 

    Если аналитически или по найденным решениям можно сделать вывод о том, 
    что про изводная $P$-го порядка
    $u^{(P)}(z)$ или даже младших порядков - не существует, то метод порядка точности $P$ - 
    избыточен. Он будет давать точность 
    меньшую, чем оправдана объёмом вычислений.
    
    От метода порядка точности $P$ мы ожидаем, что погрешность решения убывает 
    в $k^P$ раз при уменьшении шага в $k$ раз. Если этого не наблюдается на больших шагах, то
    метод избыточен и его использовать не стоит. 


    %\newpage
    %\section*{ЛР3}


\end{document}